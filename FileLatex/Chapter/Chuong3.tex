\setcounter{chapter}{2}
% Setting Chapter 3

\chapter{Thực nghiệm và kết quả thảo luận}
    \section{Kết quả phân tích \ce{^226Ra} trong đất}
    \section{Kết quả phân tích \ce{^226Ra} trong nước}
        \subsection{Đánh giá về khả năng hấp thụ mangan dioxit trên sợi vải tổng hợp}
        \subsection{Tính toán liều hiệu dụng hằng năm do uống nước chứa \ce{^222Rn}  và \ce{^226Ra}}
            Liều hiệu dụng hằng năm $D_w$ (Sv) mà một người nhận được từ uống nước chứa đồng vị \ce{^222Rn} và \ce{^226Ra} và sai số được tính toán theo công thức:  
            
            \begin{align}
                &D_w = \varepsilon. V. C_w\\
                &\sigma_{D_w} = \varepsilon.V.\sigma_{C_w} \varepsilon = \dfrac{E}{C}.\sigma_{C_w}
            \end{align}

            Trong đó: 
            \begin{itemize}
                \item $\varepsilon$ là hệ số chuyển đổi liều hiệu dụng trên một đơn vị nồng độ phóng xạ, với \ce{^222Rn}: giá trị $\varepsilon= 10^{-8}$ (Sv/Bq), trong trường hợp \ce{^226Ra} thì giá trị $\varepsilon  = 2,8 \times 10^{-7} $ (Sv/Bq)
                \item $C_w \pm \sigma_{C_w}  (Bq/m^3)  $ là nồng độ tuyệt đối của  \ce{^222Rn}  hoặc \ce{^226Ra}
                \item V là thể tích mỗi người uống trung bình một năm. Từ các số liệu thống kê, trong một ngày một người uống  uống 2 lít nước thì giá trị $V  = 2.10^{-3} \times 365 = 0,73 (m^3) $ 
            \end{itemize}
           Theo báo cáo của UNSCEAR (2000), liều hiệu dụng trung bình toàn cầu do uống nước chứa \ce{^222Rn} khoảng 2 ($\mu Sv$/năm), và nước  chứa \ce{^226Ra} khoảng 8  ($\mu Sv$/năm).  Cơ quan bảo vệ môi trường Hoa Kì, USEPA, đã đưa ra mức giới hạn nồng độ  \ce{^222Rn} và \ce{^226Ra} trong nước lần lượt là 11,1 (Bq/L) và 0,185 (Bq/L) ~\cite{Thesis:HNPThu}. 
           

    
\clearpage