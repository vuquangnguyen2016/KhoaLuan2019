\chapter*{Lời mở đầu}


Các hoạt động khai thác quặng, dầu và khí tự nhiên, các mạch nước ngầm gây ra phơi nhiễm phóng xạ tự nhiên trong môi trường. Điều này dẫn đến hàm lượng phóng xạ tại các địa điểm diễn ra các hoạt động trên có khả năng vượt ngưỡng an toàn phóng xạ cho phép. Trong số các đồng vị phóng xạ, \ce{^226Ra} được xem là một đồng vị quan trọng do có chu kỳ bán rã dài, khả năng ion hóa cao và có tính chất hóa học tương đồng với nhiều kim loại kiềm thổ khác trong đất. Các đồng vị con cháu của \ce{^226Ra} cũng có có khả năng ion hóa cao, đặc biệt \ce{^226Ra} còn sinh ra 222Rn, một đồng vị phóng xạ dạng khí có khả năng phân tán vào môi trường rất cao ~\cite{IAEANo476:revise}.

Trên thế giới, có nhiều công trình nghiên cứu cho thấy, chất thải tự nhiên do các hoạt động khai thác quặng, dầu khí,...  gây ra có hàm lượng phóng xạ rất cao trong đất, có vùng lên đến 1000 kBq/kg ~\cite{BCR:JamalAlAbdullah}. Do đó, vấn đề đặt ra cho các nhà khoa học là cần xác định mức độ ô nhiễm phóng xạ tại các khu vực này và xây dựng các quy trình chuẩn nhằm giảm thiểu tối đa phơi nhiễm phóng xạ, phục hồi tính trạng môi trường. Nhiều nhà nghiên cứu trên thế giới đã áp dụng phương pháp chiết tách phân đoạn BCR để lọc tách các kim loại nặng trong đất, nổi bật như nghiên cứu của Koguh và cộng sự (1994),  Kozuh  và cộng sự (1996), Mark D. Ho và cộng sự (1997). Phương pháp chủ yếu dựa vào tính chất hóa học, khả năng linh động của các kim loại mà xây dựng quy trình lọc tách phù hợp. Trong khóa luận, chúng tôi áp dụng quy trình chiết tách phân đoạn BCR để chiết tách \ce{^226Ra} từ pha rắn (đất) sang pha lỏng. Đây cũng được xem là một trong những kỹ thuật làm giàu \ce{^226Ra} từ các mẫu có chứa hàm lượng \ce{^226Ra} cao trong môi trường. Trong khóa luận, chúng tôi chưa có mẫu chất thải tự nhiên có hàm lượng \ce{^226Ra} cao. Vì vậy, quy trình lọc tách được áp dụng trên các mẫu đất có hàm lượng \ce{^226Ra} trên mức trung bình thế giới. 

Bên cạnh ô nhiễm phóng xạ \ce{^226Ra} trong đất, các hoạt động khai thác nhiên liệu của con người cũng làm tăng đáng kể nồng độ phóng xạ trong nước ngầm, đặc biệt là \ce{^226Ra}. Tất cả các loại nước sinh hoạt sử dụng hiện nay hầu như đều có nguồn gốc từ nước ngầm. Con người khi tiêu thụ nhiều lượng nước có chứa nồng độ \ce{^226Ra} cao có nguy cơ mắc các bệnh ung thư, đặc biệt là ung thư xương. Vì vậy, việc xử lý nhiễm bẩn phóng xạ trong các nguồn nước ngầm là vấn đề cần thiết trong thực tiễn. Trong khóa luận,  tác giả đã cải tiến phương pháp hấp thụ \ce{^226Ra} bằng sợi tổng hợp polyester tẩm \ce{MnO2} để lọc tách \ce{^226Ra} trong nước giếng dựa trên phương pháp của nhóm Willard S. Moore (1973). Phương pháp cải tiến cho kết quả khả quang: hiệu suất lọc \ce{^226Ra} trong nước từ (80-99\%), và các mẫu  đáp ứng an toàn phóng xạ, liều hiệu dụng về radon và radium đều trong ngưỡng an toàn của UNSCEAR (2000) và USEPA.


Khóa luận được chia thành ba chương:

\textbf{Chương 1}: Trình bày tổng quan về tình hình nghiên cứu trong và ngoài nước liên quan đến mục đích nghiên cứu của khóa luận.Các tính chất vật lí và hóa học của radium, tính linh động, quá trình di chuyển và phân bố của radium trong tự nhiên, ảnh hưởng của radium đến sức khỏe con người cũng như tác động đến môi trường cũng được trình bày trong chương 1.

\textbf{Chương 2}: Trình bày quy trình xác định hàm lượng \ce{^226Ra} trong đất, lọc tách radium trong đất theo quy trình BCR ba bước. Nồng độ radium trong nước được xác định bằng hệ thiết bị RAD7. Quy trình lọc tách \ce{^226Ra} trong nước giếng bằng phương pháp hấp thụ trên MnO2 ⋅ SiO2 được trình bày trong chương này.

\textbf{Chương 3}: Trình bày kết quả lọc tách \ce{^226Ra} trong đất và nước ngầm. Các đánh giá và nhận xét liên quan cũng được trình bày trong chương.

