
      Liều hiệu dụng hằng năm $D_w$ (Sv) mà một người nhận được từ uống nước chứa đồng vị \ce{^222Rn} và \ce{^226Ra} và sai số được tính toán theo công thức:  

      \begin{align}
          &D_w = \varepsilon. V. C_w\\
          &\sigma_{D_w} = \varepsilon.V.\sigma_{C_w} \varepsilon = \dfrac{E}{C}.\sigma_{C_w}
      \end{align}
      Trong đó: 

      Các hoạt động khai thác quặng, dầu và khí tự nhiên, các mạch nước ngầm gây ra phơi nhiễm phóng xạ tự nhiên trong môi trường. Điều này dẫn đến hàm lượng phóng xạ tại các địa điểm diễn ra các hoạt động trên có khả năng vượt ngưỡng an toàn phóng xạ cho phép. Trong số các đồng vị phóng xạ, 226Ra được xem là một đồng vị quan trọng do có chu kỳ bán rã dài, khả năng ion hóa cao và có tính chất hóa học tương đồng với nhiều kim loại kiềm thổ khác trong đất. Các đồng vị con cháu của 226Ra cũng có có khả năng ion hóa cao, đặc biệt 226Ra còn sinh ra 222Rn, một đồng vị phóng xạ dạng khí có khả năng phân tán vào môi trường rất cao. Trên thế giới, có nhiều công trình nghiên cứu cho thấy, chất thải tự nhiên do các hoạt động khai thác quặng, dầu khí,…gây ra có hàm lượng phóng xạ rất cao trong đất, có vùng lên đến 1000 kBq/kg [Al-Masri, 2003; Agbalagba, 2013; Al Attar, 2015; Bajoga, 2015].
      
      \begin{itemize}
          \item $\varepsilon$ là hệ số chuyển đổi liều hiệu dụng trên một đơn vị nồng độ phóng xạ, với \ce{^222Rn}: giá trị $\varepsilon= 10^{-8}$ (Sv/Bq), trong trường hợp \ce{^226Ra} thì giá trị $\varepsilon  = 2,8 \times 10^{-7} $ (Sv/Bq)
          \item $C_w \pm \sigma_{C_w}  (Bq/m^3)  $ là nồng độ tuyệt đối của  \ce{^222Rn}  hoặc \ce{^226Ra}
          \item V là thể tích mỗi người uống trung bình một năm. Từ các số liệu thống kê, trong một ngày một người uống  uống 2 lít nước thì giá trị $V  = 2.10^{-3} \times 365 = 0,73 (m^3) $ 
      \end{itemize}

     Theo báo cáo của UNSCEAR (2000), liều hiệu dụng trung bình toàn cầu do uống nước chứa \ce{^222Rn} khoảng 2 ($\mu Sv$/năm), và nước  chứa \ce{^226Ra} khoảng 8  ($\mu Sv$/năm).  Cơ quan bảo vệ môi trường Hoa Kì, USEPA, đã đưa ra mức giới hạn nồng độ  \ce{^222Rn} và \ce{^226Ra} trong nước lần lượt là 11,1 (Bq/L) và 0,185 (Bq/L) ~\cite{Thesis:HNPThu}. 
     
     