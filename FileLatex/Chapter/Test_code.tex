\chapter{CHAPTER 1}
\subsection{Phương pháp chiết tách radium trong đất}


Năm 2016, Jamal Al Abdullah cùng cộng sự thuộc Ban an toàn phóng xạ, Ủy ban năng lượng nguyên tử Syria đã thực hiện quy trình chiết tách phân đoạn BCR để chiết tách \ce{^226Ra} trong đất. Mẫu đất nhiễm phóng xạ \ce{^226Ra} được lấy từ giếng dầu tại Palmya thuộc Syria \footnotemark, hoạt độ phóng xạ \ce{^226Ra} đo được dao động từ  $1030 \pm 90$ đến $7780 \pm 530 $ (Bq/kg), giá trị trung bình là $2840 \pm 1840$ (Bq/kg). Sự khác biệt về hoạt độ \ce{^226Ra} từ các mẫu đất, được giải thích do sự khác biệt về đặc điểm của đất như độ pH, thành phần cơ giới đất \footnotemark

\footnotetext{Thành phần cơ giới đất (sa cấu đất - soil texture) đề cập đến các tỷ lệ khác nhau của ba loại hạt: cát , thịt và sét trong một loại đất. Thành phần hạt   sẽ xác định kích thước và số lượng các lỗ hỗng giữa các hạt, tại đó không khí hoặc nước bị giữ lại. Đất cát có ít các lổ hổng hơn nhưng lổ hổng lại lớn hơn đất sét, do kích thước của các hạt lớn hơn.}
\footnotetext{OK}

% \subsection{Phương pháp lọc radium trong nước}
% Năm 1975, Moore cùng cộng sự thuộc phòng thí nghiệm phân tích hải dương học Naval, Hoa Kì đã áp dụng phương pháp lọc nước nhiễm phóng xạ \ce{^226Ra} bằng sợi Mn/MnO2 ~\cite{MnO2:Moore}. Các mẫu nước được lấy các địa điểm cách vùng Felde (miền nam Texas) tại đây có nhiều nhà máy tập trung khai thác quặng uranium. Ông đã đo các mẫu nước, kết quả cho thấy nồng độ \ce{^226Ra} cao nhất đo được là $110 pCi/l$,  vượt quá ngưỡng tiêu chuẩn của Cục Tiêu chuẩn An Toàn Sức khỏe cộng đồng Hoa Kì là $3pCi/l$. Ông đã dùng sợi acrylic thấm Mn/MnO2 (phần trăm khối lượng của Mn từ 12-15\%), lọc và làm khô sợi. Thí nghiệm dùng 40g sợi Mn được lọc 2650L nước bằng hệ thống lọc nước thẩm thấu Whitley No.1 gồm 2 cột lọc và tốc độ dòng chảy từ 7.5 đến 8.21 (phút/L). Các mẫu nước sau khi lọc, lượng \ce{^226Ra} đo được là $3pCi/L$ trong ngưỡng an toàn. Kết quả đo được minh họa trong hình ~\ref{FigChapter1:Moore1975}, cho thấy rằng chỉ có khoảng 10\% nồng độ \ce{^226Ra} được lọc từ cột 2, do đó phương pháp lọc bằng bằng Mn/MnO2 phù hợp để lọc các nguồn nước mặt, nước ngầm nhiễm phóng xạ \ce{^226Ra}
% % \begin{figure}[ht]
% %     \centering
% %     \includegraphics[width=0.6\textwidth]{Image/Chapter1-Moore1975.png}
% %     \caption{Nồng độ \ce{^226Ra} đo được từ mẫu nước được lọc bằng sợi Mn. (a) Mẫu nước trước khi lọc; (b) Mẫu nước được lọc qua ở cột 1; (c) Mẫu nước (đã lọc qua từ cột 1) được tiếp tục lọc qua ở cột 2}
% %     \label{FigChapter1:Moore1975}
% % \end{figure}

% \begin{minipage}[c]{\textwidth}% to keep image and caption on one page
%     \centering
%     \includegraphics[width=6cm]{Image/Chapter1-Moore1975.png}
%     \captionof{figure}{Nồng độ \ce{^226Ra} đo được từ mẫu nước được lọc bằng sợi Mn. (a) Mẫu nước trước khi lọc; (b) Mẫu nước được lọc qua ở cột 1; (c) Mẫu nước (đã lọc qua từ cột 1) được tiếp tục lọc qua ở cột 2}
%     \label{FigChapter1:Moore1975}
% \end{minipage}

% \chapter{TEST TABLE}




% \begin{table}
%     \caption{Table with \texttt{tabularx}} \label{tab:tabularx}
    
%     \begin{tabularx}{\linewidth}{@{} l *{6}{C}@{}}
%     \toprule
%     Material & \multicolumn{6}{c@{}}{Test} \\ 
%     \cmidrule(l){2-7} 
%     & \multicolumn{2}{c}{Impact at \SI{10}{\joule}} 
%     & \multicolumn{2}{c}{Impact at \SI{14}{\joule}} 
%     & \multicolumn{2}{c}{QSI} \\ 
%     \cmidrule(lr){2-3} \cmidrule(lr){4-5} \cmidrule(l){6-7} 
%     & $F_i$ & Load drop & $F_i$ & Load drop & $F_Q$ & Load drop \\ 
%     & (\si{\kilo\newton}) & (\%) & (\si{\kilo\newton}) & (\%) & (\si{\kilo\newton}) & (\%) \\
%     \midrule
%     LTHIN\textsubscript{LVI}    & 3.24 $\pm$ 0.02       & 19.75 $\pm$ 0.55       & 3.41 $\pm$ 0.25       & 22.80 $\pm$ 1.40       & 3.32     & Multiple     \\
%     LV1\textsubscript{LVI}      & 3.19 $\pm$ 0.04       & 14.64 $\pm$ 2.55       & 3.24 $\pm$ 0.22       & 11.34 $\pm$ 4.46       & 3.10     & Multiple     \\
%     LV2\textsubscript{LVI}      & 3.51 $\pm$ 0.15       & 8.17 $\pm$ 2.36        & 3.95 $\pm$ 0.51       & 10.80 $\pm$ 1.28       & None     & None  \\ 
%     \bottomrule
%     \end{tabularx}
    
%     \bigskip\bigskip
    
%     \caption{Table with \texttt{tabular}} \label{tab:tabular}
%     \centering
%     \begin{tabular}{@{} l *{6}{c} @{}}
%     \toprule
%     Material & \multicolumn{6}{c@{}}{Test} \\ 
%     \cmidrule(l){2-7} 
%     & \multicolumn{2}{c}{Impact at \SI{10}{\joule}} 
%     & \multicolumn{2}{c}{Impact at \SI{14}{\joule}} 
%     & \multicolumn{2}{c}{QSI} \\ 
%     \cmidrule(lr){2-3} \cmidrule(lr){4-5} \cmidrule(l){6-7} 
%     & $F_i$ & Load drop & $F_i$ & Load drop & $F_Q$ & Load drop \\ 
%     & (\si{\kilo\newton}) & (\%) & (\si{\kilo\newton}) & (\%) & (\si{\kilo\newton}) & (\%) \\
%     \midrule
%     LTHIN\textsubscript{LVI}    & 3.24 $\pm$ 0.02       & 19.75 $\pm$ 0.55       & 3.41 $\pm$ 0.25       & 22.80 $\pm$ 1.40       & 3.32     & Multiple     \\
%     LV1\textsubscript{LVI}      & 3.19 $\pm$ 0.04       & 14.64 $\pm$ 2.55       & 3.24 $\pm$ 0.22       & 11.34 $\pm$ 4.46       & 3.10     & Multiple     \\
%     LV2\textsubscript{LVI}      & 3.51 $\pm$ 0.15       & 8.17 $\pm$ 2.36        & 3.95 $\pm$ 0.51       & 10.80 $\pm$ 1.28       & None     & None  \\ 
%     \bottomrule
%     \end{tabular}
    
%     \bigskip\bigskip
    
%     \caption{Table with \texttt{tabular*}} \label{tab:tabular*}
%     \setlength\tabcolsep{0pt}
%     \begin{tabular*}{\textwidth}{@{\extracolsep{\fill}} l *{6}{c}}
%         \toprule
%         Material & \multicolumn{6}{c}{Test} \\ 
%         \cmidrule(l){2-7} 
%         & \multicolumn{2}{c}{Impact at \SI{10}{\joule}} 
%         & \multicolumn{2}{c}{Impact at \SI{14}{\joule}} 
%         & \multicolumn{2}{c}{QSI} \\ 
%         \cmidrule{2-3} \cmidrule{4-5} \cmidrule{6-7} 
%         & $F_i$ & Load drop & $F_i$ & Load drop & $F_Q$ & Load drop \\ 
%         & (\si{\kilo\newton}) & (\%) & (\si{\kilo\newton}) & (\%) & (\si{\kilo\newton}) & (\%) \\
%         \midrule
%         LTHIN\textsubscript{LVI}    & 3.24 $\pm$ 0.02       & 19.75 $\pm$ 0.55       & 3.41 $\pm$ 0.25       & 22.80 $\pm$ 1.40       & 3.32     & Multiple     \\
%         LV1\textsubscript{LVI}      & 3.19 $\pm$ 0.04       & 14.64 $\pm$ 2.55       & 3.24 $\pm$ 0.22       & 11.34 $\pm$ 4.46       & 3.10     & Multiple     \\
%         LV2\textsubscript{LVI}      & 3.51 $\pm$ 0.15       & 8.17 $\pm$ 2.36        & 3.95 $\pm$ 0.51       & 10.80 $\pm$ 1.28       & None     & None  \\ 
%         \bottomrule
%     \end{tabular*}

% \end{table}




% \begin{table}
%     \caption{Table with \texttt{tabular*}} \label{tab:tabular*}
%     % \setlength\tabcolsep{0pt}
%     \begin{tabular*}{\textwidth}{@{\extracolsep{\fill}}  *{7}{c}}
%         \toprule
%         Material & \multicolumn{6}{c}{Test} \\ 
%         \cmidrule{2-7} 
%         & \multicolumn{2}{c}{Impact at \SI{10}{\joule}} 
%         & \multicolumn{2}{c}{Impact at \SI{14}{\joule}} 
%         & \multicolumn{2}{c}{QSI} \\ 
%         \cmidrule{2-3} \cmidrule{4-5} \cmidrule{6-7} 
%         & $F_i$ & Load drop & $F_i$ & Load drop & $F_Q$ & Load drop   \\ 
%         & (\si{\kilo\newton}) & (\%) & (\si{\kilo\newton}) & (\%) & (\si{\kilo\newton}) & (\%) \\
%         \midrule
%         LTHIN\textsubscript{LVI}    & 3.24 $\pm$ 0.02       & 19.75 $\pm$ 0.55       & 3.41 $\pm$ 0.25       & 22.80 $\pm$ 1.40       & 3.32     & Multiple      \\
%         LV1\textsubscript{LVI}      & 3.19 $\pm$ 0.04       & 14.64 $\pm$ 2.55       & 3.24 $\pm$ 0.22       & 11.34 $\pm$ 4.46       & 3.10     & Multiple     \\
%         LV2\textsubscript{LVI}      & 3.51 $\pm$ 0.15       & 8.17 $\pm$ 2.36        & 3.95 $\pm$ 0.51       & 10.80 $\pm$ 1.28       & None     & None     \\ 
%         LV2\textsubscript{LVI}      & 3.51 $\pm$ 0.15       & 8.17 $\pm$ 2.36        & 3.95 $\pm$ 0.51       & 10.80 $\pm$ 1.28       & None     & None     \\ 
%         LV2\textsubscript{LVI}      & 3.51 $\pm$ 0.15       & 8.17 $\pm$ 2.36        & 3.95 $\pm$ 0.51       & 10.80 $\pm$ 1.28       & None     & None     \\ 
%         LV2\textsubscript{LVI}      & 3.51 $\pm$ 0.15       & 8.17 $\pm$ 2.36        & 3.95 $\pm$ 0.51       & 10.80 $\pm$ 1.28       & None     & None     \\ 
%         LV2\textsubscript{LVI}      & 3.51 $\pm$ 0.15       & 8.17 $\pm$ 2.36        & 3.95 $\pm$ 0.51       & 10.80 $\pm$ 1.28       & None     & None     \\ 
%         LV2\textsubscript{LVI}      & 3.51 $\pm$ 0.15       & 8.17 $\pm$ 2.36        & 3.95 $\pm$ 0.51       & 10.80 $\pm$ 1.28       & None     & None     \\ 
%         \bottomrule
%     \end{tabular*}
% \end{table}




% \chapter[Quy trình chiết tách \ce{^226Ra}]{Quy trình chiết tách \ce{^226Ra} trong đất và nước}
% \chaptermark{Quy trình chiết tách \ce{^226Ra}}
% {
%     \section{OK}
%     \subsection{OK2}
%     \subsubsection{ok3}
%     skd

%     Tuttavia, resta da determinare la "giusta" trasformazione di stato $T(x)$. A tal fine si introducono, brevemente, alcuni importanti strumenti matematici.

%     \begin{tabular}{llll}
%         \hline
%         test entry & test entry & test entry & test entry \\
%         test entry & test entry & test entry & test entry \\
%         test entry & test entry & test entry & test entry \\
%         \hline
%       \end{tabular}
      
% % \theoremstyle{plain}
% % \newtheorem*{Frobenius}{Teorema di Frobenius}
% % \begin{Frobenius}
% %     Sia $\left\lbrace v_{1},v_{2},\dots,v_{n}\right\rbrace$ un insieme di campi vettoriali linearmente indipendenti. L'insieme è completamente integrabile se, e solo se, esso è involutivo.
% % \end{Frobenius}

% % \newtheorem*{FondTheorem}{Teorema}
% % \begin{FondTheorem}
% %     Il sistema non lineare
% %     \begin{equation*}
% %         \begin{cases}
% %         &\dot{x} = f(x) + g(x)u\\
% %         &y=h(x)
% %         \end{cases}
% %     \end{equation*}
% %     dove $f$ e $g$ sono campi vettoriali di classe $C^{\infty}$, si dice input-state linearizzabile se, e solo se, esiste una regione $\Omega$ tale che:
%     \begin{itemize}
%         \item i campi vettoriali $\left\lbrace g,ad_{f} \; g, \dots, ad_{f}^{n-1} \; g\right\rbrace$ sono linearmente indipendenti in $\Omega$
%         \item l'insieme $\left\lbrace g,ad_{f} \; g, \dots, ad_{f}^{n-2} \; g\right\rbrace$ è involutivo
%     \end{itemize}
% \end{FondTheorem}
    	    
%     Những khám phá đầu tiên về Radium có liên hệ mật thiết với các nghiên cứu về phóng xạ của các vật liệu tự nhiên. Trong nghiên cứu của Marie Curie vê quặng uranium, bà phát hiện rằng quặng tự nhiên uranium pitchblende có hoạt độ phóng xạ lớn hơn rất nhiều so với muối uranium tinh khiết. Bà cùng chồng đã chiết tách, phân tích một lượng lớn quặng uranium, 

%     In my text, I let make the paragraph indent automatically by latex. That means everywhere I want to have a paragraph indentation in my text, I just press the enter key twice. However, when I look at the PDF version, I see that the line spacing above and below the paragraph input lines are different. Do you know how to fix this? I circled it red in the picture.

%     In my text, I let make the paragraph indent automatically by latex. That means everywhere I want to have a paragraph indentation in my text, I just press the enter key twice. However, when I look at the PDF version, I see that the line spacing above and below the paragraph input lines are different. Do you know how to fix this? I circled it red in the picture.

%     In my text, I let make the paragraph indent automatically by latex. That means everywhere I want to have a paragraph indentation in my text, I just press the enter key twice. However, when I look at the PDF version, I see that the line spacing above and below the paragraph input lines are different. Do you know how to fix this? I circled it red in the picture.

%     In my text, I let make the paragraph indent automatically by latex. That means everywhere I want to have a paragraph indentation in my text, I just press the enter key twice. However, when I look at the PDF version, I see that the line spacing above and below the paragraph input lines are different. Do you know how to fix this? I circled it red in the picture.

%     \begin{itemize}
%         \item     In my text, I let make the paragraph indent automatically by latex. That means everywhere I want to have a paragraph indentation in my text, I just press the enter key twice. However, when I look at the PDF version, I see that the line spacing above and below the paragraph input lines are different. Do you know how to fix this? I circled it red in the picture.
%         \item     In my text, I let make the paragraph indent automatically by latex. That means everywhere I want to have a paragraph indentation in my text, I just press the enter key twice. However, when I look at the PDF version, I see that the line spacing above and below the paragraph input lines are different. Do you know how to fix this? I circled it red in the picture.

%         \item     In my text, I let make the paragraph indent automatically by latex. That means everywhere I want to have a paragraph indentation in my text, I just press the enter key twice. However, when I look at the PDF version, I see that the line spacing above and below the paragraph input lines are different. Do you know how to fix this? I circled it red in the picture.

%         \item     In my text, I let make the paragraph indent automatically by latex. That means everywhere I want to have a paragraph indentation in my text, I just press the enter key twice. However, when I look at the PDF version, I see that the line spacing above and below the paragraph input lines are different. Do you know how to fix this? I circled it red in the picture.
 
%     \end{itemize}

%     họ phát hiện được rằng chắn chắn tồn tại hai nguyên tố phóng xạ mới. Vào năm 1898, họ tuyên bố khám phá được nguyên tố phóng xạ mới đặt tên là polonium (đặt tên theo quê hương của Marie Curie), và một nguyên tố phóng xạ kì lạ mà hoạt độ phóng của nó gấp 900 lần uranium (sau này phát hiện là hơn một triệu lần), tính chất hóa học tương tự như barium, gọi tên là radium, (dựa theo thuật ngữ 'phóng xạ' - 'radioactivity', phân rã tự nhiên của vật chất). 
%     \subsection{Ok}
%     Năm 1903, vợ chồng bà cũng Henri Becquerel nhận giải Nobel Prize vật lí cho khám phá hoạt độ phóng xạ. Sau những khám phá đầu tiên về radium, bà đã làm việc vật vả trong thời gian dài để nghiên cứu chiết tách Radium từ hàng tấn quặng uranium (khai thác từ vùng St. Joachimsthal, sau này là Jachymov ở Czechoslovakia)trong phòng thí nghiệm nhỏ ở ngoại ô Paris.  Vào năm 1910, bà cùng André-Louis Debierne, đã áp dụng phương pháp điện phân Radium chloride đã sử dụng thuỷ ngân làm điện cực cathode, sau đó dùng nhiệt để chưng cất thuỷ ngân và có được Radium tinh khiết. Trong năm 1911, Bà nhận được giải Nobel hóa học cho các nghiên cứu về polonium, và các tính chất vật lí của radium , tính chất hóa học của hợp chất radium, quy trình chiết tách radium tinh khiết. Bà là người phụ nữ đâu tiên trên thế giới nhận giải thương danh giá Nobel và hai lần được giải thưởng này cho cống hiến về khoa học. \cite[tr.3-4]{Ra:revise}
%             \begin{equation}
%                 A_{\ce{^232Th}} = A_{\ce{^228Ra}} = A_{\ce{^228Th}} = A_{\ce{^224Ra}} = \cdots   
%             \end{equation}

%         Một đặc điểm\footnote{OK} nữa của cân bằng thể kỉ của chuỗi là hàm lượng tự nhiên của \footnote{OK} đồng vị mẹ rất cao nhưng đồng vị của các đồng vị con cháu tương đối nhỏ. Ví dụ, trong chuỗi của Thorium: Tỉ lệ mole của \ce{^228Ra} và \ce{^232Th} khi cân bằng thế kỉ \footnote{OK}
%             \begin{equation}
%                     \dfrac{N_{\ce{^228Ra}}}{N_{\ce{^232Th}}} = \dfrac{T_{1/2}^{228Ra}}{T_{1/2}^{232Th}} = 4.0 \times 10^{-9}
%                 \end{equation}
%             Tương tự, trong chuỗi \ce{^238U} khi xảy ra cân bằng thế kỉ, tỉ lệ mole của \ce{^226Ra} và \ce{^238U} là: 
%                 \begin{equation}
%                     \dfrac{N_{\ce{^226Ra}}}{N_{\ce{^238U}}} = \dfrac{T_{1/2-226Ra}}{T_{1/2,238U}} = 7.1\times 10^{-13}
%                 \end{equation}      
    
% }

% \chapter{Test Numbered Chapter}
% \section{Ok}
% \subsection{OKw}
% \subsubsection{OK}
% {
%     \begin{equation}
%         \dfrac{1}{2} = \sin(e^x)
%     \end{equation}
% }
% {
%     % \documentclass[10pt,a4paper]{article}
%     % \usepackage[demo]{graphicx}
%     % \usepackage{subfig}
%     % \begin{document}
%     % \begin{figure}%
%     %     \centering
%     %     \subfloat[label 1]{{\includegraphics[width=5cm]{img1} }}%
%     %     \qquad
%     %     \subfloat[label 2]{{\includegraphics[width=5cm]{img2} }}%
%     %     \caption{2 Figures side by side}%
    %     \label{fig:example}%
    % \end{figure}
    % \end{document}



    % \begin{table}
    
    %     \caption{Quy trình chiết tách ba bước BCR (tham khảo Rauet et al,2000)}
    %     \begin{center}
    %         \scalebox{0.76}{
    %         \begin{tabular}{l  p{0.4\textwidth} p{0.85\textwidth}}
    %             \hline
    %             Bước & Phân đoạn & Dung dịch và cách thực hiện\\ \hline
    %             1    & F1: Chiết li axit, liên kết cacbonat & 0.11M \ce{CH3COOH}, 40mL, nhiệt độ phòng, lắc trong 16h, li tâm 3000g trong 20 phút \\
    %             2    & F2: Khử, liên kết Fe-Mn oxit & 0.5M \ce{NH4OH.HCl} (pH=2,\ce{HNO3}), 40mL, nhiệt độ phòng, lắc trong 16h, li tâm 3000g trong 20 phút  \\
    %             3    & F3: Oxi hóa, liên kết hữu cơ và sulfua & 30\% \ce{H2O2} 10mL,  nhiệt độ phòng, sau đó, 1h, nhiệt độ $85 \pm 2 ^\circ C$; Thêm tiếp 30\% \ce{H2O2} 10mL, nhiệt độ $85 \pm 2 ^\circ C$, trong 1h; thêm 1M \ce{CH3COONH4} 50mL (pH=2, \ce{HNO3}), nhiệt độ phòng, lắc trong 16h, li tâm 3000g trong 20 phút \\
    %             4    & F4: Hòa tan dư lượng & 3mL nước cất, 6M \ce{HCl} 75mL, 14M \ce{HNO3} 25mL để qua đêm, nhiệt độ phòng; đun sôi trong 2h, làm lạnh nhanh, 20mL nước cất lắc 15 phút, li tâm 3000g trong 20 phút\\
    %             \hline
    %         \end{tabular}
    %     }
    %     \end{center}
    % \end{table}


    % Ask Question
    % 283

    % This question already has an answer here:

    %     Two figures side by side 4 answers

    % I want to place 2 images side by side in LaTeX. I have 2 .png files and I don't understand how to do it in LaTeX. I have tried many ways but could not get a good result.
    % graphics floats positioning
    % shareimprove this question
    % edited Aug 7 '16 at 14:43
    % RAnders00
    % 746
    % asked Dec 8 '11 at 11:33
    % nikhil
    % marked as duplicate by Ignasi, user13907, Jesse, egreg, Masroor Jan 28 '15 at 10:52

    % This question has been asked before and already has an answer. If those answers do not fully address your question, please ask a new question.
    % migrated from stackoverflow.com Dec 8 '11 at 14:23

    % This question came from our site for professional and enthusiast programmers.

    %     1
    %     Welcome to TeX.sx! Your question was migrated here from Stack Overflow. Please register on this site, too, and make sure that both accounts are associated with each other, otherwise you won't be able to comment on or accept answers or edit your question. – Torbjørn T. Dec 8 '11 at 14:25
    %     2
    %     This answer helped me: tex.stackexchange.com/a/83665/61609 – ricardoramos Aug 28 '17 at 16:01

    % add a comment
    % 4 Answers
    % active
    % oldest
    % votes
    % 374

    % For two independent side-by-side figures, you can use two minipages inside a figure enviroment; for two subfigures, I would recommend the subcaption package with its subfigure environment; here's an example showing both approaches:

    % % \documentclass{article}
    % % \usepackage[demo]{graphicx}
    % % \usepackage{caption}
    % % \usepackage{subcaption}

    % % \begin{document}

    % % \begin{figure}
    % % \centering
    % % \begin{subfigure}{.5\textwidth}
    % %   \centering
    % %   \includegraphics[width=.4\linewidth]{image1}
}
%   \caption{A subfigure}
%   \label{fig:sub1}
% \end{subfigure}%
% \begin{subfigure}{.5\textwidth}
%   \centering
%   \includegraphics[width=.4\linewidth]{image1}
%   \caption{A subfigure}
%   \label{fig:sub2}
% \end{subfigure}
% \caption{A figure with two subfigures}
% \label{fig:test}
% \end{figure}

% \begin{figure}
% \centering
% \begin{minipage}{.5\textwidth}
%   \centering
%   \includegraphics[width=.4\linewidth]{image1}
%   \captionof{figure}{A figure}
%   \label{fig:test1}
% \end{minipage}%
% \begin{minipage}{.5\textwidth}
%   \centering
%   \includegraphics[width=.4\linewidth]{image1}
%   \captionof{figure}{Another figure}
%   \label{fig:test2}
% \end{minipage}
% \end{figure}

% \end{document}


% $\alpha$: Phân rã alpha\\
% $\beta$: Phân rã beta\\

% \fbox{\begin{minipage}{5em}
%     \begin{center}
%       \ce{^238U}\\
%         \small 4,47 tỷ năm
%     \end{center}
%   \end{minipage}}

% \vspace{1em}

%   \fbox{\begin{minipage}{5em}
%     \begin{center}
%       \ce{^234Th}\\
%         \small 24,10 ngày
%     \end{center}
%   \end{minipage}}
  
% \vspace{1em}

%   \fbox{\begin{minipage}{5em}
%     \begin{center}
%       \ce{^234Pa}\\
%         \small 6,70 giờ
%     \end{center}
%   \end{minipage}}
  
% \vspace{1em}

%   \fbox{\begin{minipage}{5em}
%     \begin{center}
%       \ce{^234U}\\
%         \small 245.5 nghìn năm
%     \end{center}
%   \end{minipage}}

%   \vspace{1em}


  
%   \fbox{\begin{minipage}{5em}
%     \begin{center}
%       \ce{^230Th}\\
%         \small 75,38 nghìn năm
%     \end{center}
%   \end{minipage}}
%   \vspace{1em}

  
%   \fbox{\begin{minipage}{5em}
%     \begin{center}
%       \ce{^226Ra}\\
%         \small 1602 năm
%     \end{center}
%   \end{minipage}}
%   \vspace{1em}

  
%   \fbox{\begin{minipage}{5em}
%     \begin{center}
%       \ce{^222Rn}\\
%         \small 3,824 ngày
%     \end{center}
%   \end{minipage}}
%   \vspace{1em}
  
%   \fbox{\begin{minipage}{5em}
%     \begin{center}
%       \ce{^222Rn}\\
%         \small 3,824 ngày
%     \end{center}
%   \end{minipage}}
%   \vspace{1em}
  
%   \fbox{\begin{minipage}{5em}
%     \begin{center}
%       \ce{^218Po}\\
%         \small 3,10 phút
%     \end{center}
%   \end{minipage}}
%   \vspace{1em}
  
%   \fbox{\begin{minipage}{5em}
%     \begin{center}
%       \ce{^214Pb}\\
%         \small 26,8 phút
%     \end{center}
%   \end{minipage}}
%   \vspace{1em}
  
%   \fbox{\begin{minipage}{5em}
%     \begin{center}
%       \ce{^214Bi}\\
%         \small 19.9 phút
%     \end{center}
%   \end{minipage}}
%   \vspace{1em}
  
%   \fbox{\begin{minipage}{5em}
%     \begin{center}
%       \ce{^214Po}\\
%         \small 164 $\mu s$
%     \end{center}
%   \end{minipage}}
%   \vspace{1em}
    
%     \fbox{\begin{minipage}{5em}
%       \begin{center}
%         \ce{^210Pb}\\
%           \small 22,3 năm 
%       \end{center}
%     \end{minipage}}
%     \vspace{1em}
    
%     \fbox{\begin{minipage}{5em}
%       \begin{center}
%         \ce{^210Bi}\\
%           \small 5.01 ngày 
%       \end{center}
%     \end{minipage}}
%     \vspace{1em}
    
%     \fbox{\begin{minipage}{5em}
%       \begin{center}
%         \ce{^210Po}\\
%           \small 138,4 ngày 
%       \end{center}
%     \end{minipage}}
%     \vspace{1em}
    
%     \fbox{\begin{minipage}{5em}
%       \begin{center}
%         \ce{^206Pb}\\
%         bền  
%       \end{center}
%     \end{minipage}}
%     \vspace{1em}
%  % Test 1       ~\cite{BenhAnLichSu}: online

% % Test 2       ~\cite{Thesis:HNPThu}: Thesis 1

% % Test 3       ~\cite{Thesis:NTHien}: Thesis 2

% % Test 4      ~\cite{Ra:revise}: English


% % Ki hieu footnote cong thuc toan 
% % \begin{eqnarray}
% %   foo\footnotemark
% % \end{eqnarray}

% % \begin{table}[t]
% %   \caption{Caption.}
% %   \label{table:label}
%   \begin{tabularx}{\columnwidth}{| Y | Y | *{2}{ >{\RaggedRight\arraybackslash}X |}}
%       \hline
%        & OA & Acc & Time [minutes] \\
%       \hline
%       \multirow{2}{*}{A} & \multirow{2}{*}{RowB} & ROWC1 & ROWD1 \\
%                                                 \cline{3-4}
%                         &                        & RowC2 & ROWD2 \\ 
%       \hline
%       \multirow{2}{*}{A} & B1  & ROWC1 & ROWD1 \\
%                          \cline{2-4}
%                         &  B2  & RowC2 & ROWD2 \\ 
      
%       \hline
%   \end{tabularx}
% \end{table}

% \begin{table}[t]
%   \caption{Caption.}
%   \label{table:label}
%   \begin{tabularx}{\columnwidth}{| Y | Y | *{2}{ >{\RaggedRight\arraybackslash}X |}}
%       \hline
%        & OA & Acc & Time [minutes] \\
%       \hline
%       \multirow{2}{*}{A} & \multirow{2}{*}{RowB} & ROWC1 & ROWD1 \\
%                                                 \cline{3-4}
%                         &                        & RowC2 & ROWD2 \\ 
%       \hline
%       \multirow{2}{*}{A} & B1  & ROWC1 & ROWD1 \\
%                          \cline{2-4}
%                         &  B2  & RowC2 & ROWD2 \\ 
      
%       \hline
%   \end{tabularx}
% \end{table}


% \begin{table}
%   \begin{tabular}{|L|c|L|}\hline
%   one & two & three \\\hline
%   This is two line thing and centered & only one line&  \multicolumn{1}{m{3cm}|}{This is justified and may go to second line as well, neatly}\\\hline
%     one & two & three \\\hline
%     one & two & three \\\hline
%   \end{tabular}
%   \end{table}



% \footnotetext{This answer depends on your document class and some related packages. if you are using a standard class you can use the package}

% Những giai điệu, những hình ảnh ... sao mà thân quen đến thế. Đón tết đến, xuân về với những giai điệu nhẹ nhàng, bình yên và đầy rạo rực với "Những Bài Hát Hay" qua album "Những bản nhạc Xuân không lời - Nhẹ nhàng, thư thái cho ngày đầu năm", những giai điệu mộc mạc và thân quen của quê hương Việt Nam, của truyền thống dân tộc Việt Nam. \footnotemark


% \footnotetext{Hay qua ha}   before the next footnote\footnote{the nextone}



% \begin{table} 
%   \begin{tabular}{|p{4cm}|p{4cm}|} 
%   \hline \textbf{Heading 1} 
  
%     Text col 1 with plenty of extra text.
  
%     And more lines of text in column one.
  
%     Yet more lines.
%     &\textbf{Heading 2} 
  
%       Text col 2 with plenty of extra text.\\ 
%   \hline 
%   \end{tabular} 
%   \end{table}

%   \begin{table}[h]
%     \centering
%     \begin{tabular}{>{\centering\arraybackslash}m{2.7cm}|>{\centering\arraybackslash}m{3.7cm}|>{\centering\arraybackslash}m{3.7cm}}
%     \hline
%     Only one line here. & Same for column two. & Same for column three. \\
%     \hline
%     We have two lines here, each line centered. & This cell will have three lines and each line centered as well. & Here is a justified cell, please check this out! \\
%     \hline
%     This cell has one line. & So does this one. & And this one, too! \\
%     \hline
%     \end{tabular}
%     \end{table}


%   %   \begin{table}

%   %     \centering
  %     \caption{Các đồng vị của Radium}
  %     \begin{tabular}{>{\centering\arraybackslash}m{1.5cm} >{\centering\arraybackslash}m{2.5cm}>\centering\arraybackslash}m{2.7cm} >{\centering\arraybackslash}m{2.7cm}}
  %     \hline
  %     Đồng vị &   Thời gian bán rã    &   Loại phân rã, Năng lượng (MeV) &  Hoạt độ riêng (Bq/g) \\ 
  %     % \hline
  %     % \ce{^223Ra}     & 11.43 ngày    &   Phân rã $\alpha$ &  $1.896 \times 10^{15}$ \\
  %     % \ce{^224Ra}     & 3.623 ngày    &   Phân rã $\alpha$ &  $5.92\times 10^{15}$ \\
  %     % \ce{^226Ra}     & 1600 năm    &   Phân rã $\alpha$ &  $3.66$ \\
  %     % \ce{^228Ra}     & 15.75 năm    &   Phân rã $\beta$ &  $1.0 \times 10^{13}$\\
  %     % \hline
  %     % \multirow{4}{*}{\ce{^223Ra}}  & \multirow{4}{*}{11.43 ngày} & $\alpha_3, 5.745(9.1\%)$ & multirow{4}{*}{$1.896 \times 10^{15}$}\\
  %     %                               &                             & $\alpha_4, 5.714(53.7\%)$ &  \\
  %     %                               &                             & $\alpha_5, 5.605(26.0\%)$ & \\
  %     %                               &                             & $\alpha_6, 5.538(9.1\%)$ &  \\
  %     \end{tabular}
  % \end{table}


  % \begin{table}[htbp]
  %   \centering
  %   \caption{Các đồng vị của Ra trong tự nhiên}
  %   \begin{tabular}{>{\centering\arraybackslash}m{2.0cm} >{\centering\arraybackslash}m{2.cm} >{\centering\arraybackslash}m{4.cm} >{\centering\arraybackslash}m{3.0cm}}
  %   \hline
  %   Đồng vị  & Thời gian bán rã & Loại phân rã, năng lượng (MeV) & Hoạt độ riêng (Bq/g)\\
  %   \hline
  %   \multirow{4}{*}{\ce{^223Ra}} & \multirow{4}{*}{11,43 ngày } & $\alpha_3; 5,745(9,1\%)$ & \multirow{4}{*}{$1,896 \times 10^{15}$} \\
  %                       &                       & $\alpha_4; 5,714(53,7\%)$& \\ 
  %                       &                       & $\alpha_5; 5,605(26,0\%)$ & \\ 
  %                       &                       & $\alpha_6 5,538(9,1\%)$ & \\ 
  %   \hline
  %   \multirow{2}{*}{\ce{^224Ra}} & \multirow{2}{*}{ 3,632 ngày } & $\alpha_0; 5,685(94,9\%)$ & \multirow{2}{*}{$5,92 \times 10^{15}$} \\
  %                       &                       & $\alpha_1; 5,685(5,1\%)$& \\ 
  %   \hline
  %   \multirow{2}{*}{\ce{^226Ra}} & \multirow{2}{*}{ 1600 năm } & $\alpha_0; 4,784(94,55\%)$ & \multirow{2}{*}{$3,66$} \\
  %                       &                       & $\alpha_1; 4,601(5,45\%)$& \\ 
  %   \hline
  %   \ce{^228Ra}         &         5,75 năm &      $\beta; 0.046$      & $1,0 \times 10^{13}$\\
  %   \hline
  %   \end{tabular}
  %   \end{table}

% \begin{table}[htb]
%     \resizebox{\textwidth}{!}
%     {% use resizebox with textwidth
%           \begin{tabular}{ l | l }
%             {\bf Symptom} & {\bf Metric} \\
%           \hline
%           Class that has many accessor methods and accesses a lot of external data & ATFD is more than a few\\
%           Class that is large and complex & WMC is high\\
%           Class that has a lot of methods that only operate on a proper subset of the instance variable set & TCC is low\\
%           \end{tabular}% close resizebox
%     }
%         \end{table}
        
%     \begin{table}[htb]
%       \resizebox{\textwidth}{!}
%       {
        
%           \begin{tabular}{ >{\centering\arraybackslash}m{0.8in}  >{\centering\arraybackslash}m{0.8in} >{\centering\arraybackslash}m{0.8in} >{\centering\arraybackslash}m{0.8in} >{\centering\arraybackslash}m{.75in} >{\centering\arraybackslash}m{.75in} >{\centering\arraybackslash}m{.7in}}
%             \toprule[1.5pt]
%             {\bf Test} & {\bf Regression 1} & {\bf Mean} & {\bf Std. Dev} & {\bf Min} & {\bf Max} & {\bf Test}\\ 
%             \midrule
%             text vu quang nguyen  & Le quang      &  text     & text      &  text     &  text     &text\\
%             text vu quang nguyen  & Le quang      &  text     & text      &  text     &  text     &text\\
%             \bottomrule[1.25pt]
%           \end {tabular}
        
%       }
%     \end{table}


%          \begin {table}[!htbp]
%             \caption {Table Title} \label{tab:title} 
%             \begin{center}
%                 \begin{tabular}{ >{\centering\arraybackslash}m{0.8in}  >{\centering\arraybackslash}m{0.8in} >{\centering\arraybackslash}m{0.8in} >{\centering\arraybackslash}m{0.8in} >{\centering\arraybackslash}m{.75in} >{\centering\arraybackslash}m{.75in} >{\centering\arraybackslash}m{.7in}}
%                 \toprule[1.5pt]
%                 {\bf Test} & {\bf Regression 1} & {\bf Mean} & {\bf Std. Dev} & {\bf Min} & {\bf Max} & {\bf Test}\\ 
%                 \midrule
%                 text vu quang nguyen  & Le quang      &  text     & text      &  text     &  text     &text\\
%                 text vu quang nguyen  & Le quang      &  text     & text      &  text     &  text     &text\\
%                 \bottomrule[1.25pt]
%                 \end {tabular}
%             \end{center}
%         \end {table}

%         \DoubleSpacing
%         \frontmatter
%         \pagestyle{toc}
%         \tableofcontents
%         \listoftables
%         \listoffigures

%         \mainmatter
        
%         \pagestyle{plain} 
        
